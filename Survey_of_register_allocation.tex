\documentclass[12pt]{book}
\usepackage{hyperref}
\hypersetup
{
  colorlinks,
  citecolor=black,
  filecolor=black,
  linkcolor=black,
  urlcolor=black
}
\title{Severy of Register Allocation}
\author{Jian-Ping Zeng}
\begin{document}

\maketitle
\tableofcontents
\chapter{Preface}
With rapidly development of CPU speed, the gap of speed between CPU and memory
greatly increasing from last century to nowadays.

\chapter{Introduction}

\chapter{Local method}
\section{Simply based on cache on Top of Stack}

\chapter{Global method}

\section{Graph coloring}
This method is the first globally approach invented by Chainth etc in 80s, 20
century for IBM 360 serial of super computer. It would obtain significantly
excellent allocation result for RISC architecture. Generally, it uses a
classical graph coloring to model register allocation, so that after solving the
coloration, we can map the coloring result into register assignment.

\section{Linear Scan}
As we illustrated in the last chapter, the solution to register allocation based
on coloring has it's intrinsic shortege that it is needed to spend much time on
register allocation of compiler in order to obtain the optimial or near-optimial
allocation performance. However, Just in time(JIT) compiler usually tradeoff compilation
time for execution of generated machine code, in this case, people generally
think it is important to short compiling time than code quality.

\section{PBQP}
PBQP means that partitioned boolean quadratic problem which is a non-liearly
optimization problem in mathmatic.

\section{Integral Linear Programming}
By transforming the register allocation into integral programming as follows,
allocation would be solved in optimal.

\section{Multi-commodity Network flow}
Also, there are many researchers conducts another method that maps allocation to
a multi-commodity network flow(MCNF) problem. Then, transforming the solution of MCNF
to allocation result.

\section{Puzzle solution}
TODO

\chapter{Conslusion}
In this book, we conduct a severy for register allocation, illustrating several
important strategies used for solving register allocation problem in detail and
making a comparison about those methods. Local allocation is pretty faster than
all of other ways at the cost of generated code quality.

\end{document}
